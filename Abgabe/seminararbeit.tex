\documentclass[runningheads,a4paper]{llncs}
\usepackage{amssymb}
\usepackage{url}
\usepackage{times}
\usepackage{float}
\usepackage[T1]{fontenc}
\usepackage{graphicx}
\usepackage{color}
\usepackage{soul}  
\usepackage{nameref}  
\usepackage{amsbsy}  
\usepackage{bezier}  
\usepackage{colortbl}  
\usepackage[leqno,fleqn]{amsmath}  
\usepackage{verbatim}
\usepackage{listings}
\usepackage{qtree}
% deutsche Silbentrennung
\usepackage[ngerman]{babel}
% wegen deutschen Umlauten
\usepackage[utf8]{inputenc}

\setcounter{tocdepth}{3}
\newcommand{\keywords}[1]{\par\addvspace\baselineskip
\noindent\keywordname\enspace\ignorespaces#1}

\begin{document}
\mainmatter

\title{Software Projekt Anwendungen von Algorithmen}
\subtitle{Metaheuristiken II}
\date{Wintersemester 2014/2015}

\author{Robert Gottwald, Lars Parmakerli, Phil Schmidt}

\institute{
Freie Universit\"at Berlin\\ Institut for Computer Science \\ robert.gottwald@fu-berlin.de \\ -Lars - fu-Adresse-\\  philschmidt@inf.fu-berlin.de \\
\url{https://www.inf.fu-berlin.de/lehre/WS14/SWPAlg/index.html}
}
\maketitle
\begin{abstract}
Der folgende Text befasst sich mit der Arbeit der Gruppe Metaheuristiken II des Softwareprojekts Anwendungen von Algorithmen im Wintersemester 2014/15. Dabei werden wir uns mit zwei  spezifischen Metaheuristiken befassen, namentlich Simulated Annealing und ein genetisch-inspiriertes Verfahren, sowie mit dem globalen Lösungsverfahren Branch-and-Bound und seinen Schwierigkeiten für die Anwendung auf unsere Probleme und schließlich zuletzt mit nicht-linearen Optimierungsverfahren unter Zuhilfenahme numerischer Verfahren bezogen auf die gewählten Aufgabenstellungen de Stapelns und des Packens.

\end{abstract}

\section{Metaheuristiken}

Das Wort Metaheuristik setzt sich aus den beiden wörtern 'Meta' und 'Heuristik'  zusammen.
Eine Heuristik ist ein Verfahren, welches für eine bestimmte Art von Problemen eine Lösung findet, die als ausreichend 'gut' erachtet wird. 
Das Wort 'Meta-' bedeutet in diesem Kontext 'auf einer Höheren Ebene'.

Eine Metaheuristik ist demnach möglichst effizientes Lösungsverfahren, welches sich auf einer höher abstahierten Ebene abspielt und somit kein Wissen über das unterliegende Problem benötigt. Die einzige bestehende Voraussetzung für das Anwenden einer Metaheuristik auf ein System, ist eine Funktion existiert, welche die Güte einer Lösung berechnen kann.

Metaheuristiken stellen nützliche Werkzeuge dar, um Optimierungsproleme in komplexeren Systemen zu lösen, ohne für diese eine spezifische, kompliziertere Heuristik zu entwerfen, sind also sozusagen wiederverwendbare Lösungswege. Der Nachteil von Metaheuristiken liegt in der Effizienz in Laufzeit, sowie der meist niedrigeren Güte der erzielten Lösung.

In unserem Softwareprojekt bestehen Metaheuristikmodule aus generischen Klassen mit einem Typenparameter für die Anwendungsdomäne, sowie mindestens einer Auswertungsfunktion. Einzelne Implementierungen werden in den Folgekapiteln beschrieben.


\section{Simulated Annealing}

\subsection{Inspiration}

Metaheuristiken liegen zumeist realen Prozessen zugrunde, welche ein bestimmtes verhalten aufweisen. Simulated Annealing ist vom namensgebenden Annealing-Prozess aus der Schwerindustrie inspiriert.

Beim Annealing wird Metal Metall stark erhitzt und über einen längeren Zeitraum langsam abgekühlt. Bei den hohen Temperaturen tritt eine starke Molekular-Bewegung im zu bearbeitenden Material auf. Dadurch, dass der Abkühlungsprozess über einen längeren Zeitraum gestreckt wird, reduziert sich die Teilchenbewegung auch nur allmählich. Der Abkühlungsprozess wird so gesteuert durch etwaige Behandlungsmethoden, dass die einzelnen Teilchen einen für einen bestimmten zu erzielenden Effekt, wie zum Beispiel eine stabile Statik, guten Zustand einnehmen.

Wie der Name uns sagt, ist die Herangehensweise beim Simulated Annealing nun so, dass wir einen derartigen physikalischen Prozess über einen abstrakten Wertebereich simulieren.

\subsection{Simulation}



\section{Genetische Heuristik}

\section{Branch and Bound}

\section{Nicht-lineare Optimierung mit numerischen Verfahren des Stapelproblems}

\section{Nicht-lineare Optimierung mit numerischen Verfahren des Packenproblems}

\end{document}

